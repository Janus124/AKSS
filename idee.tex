Trusted Execution Environments (TEEs) haben sich als wichtige und notwendige Idee im Bereich der Computersicherheit etabliert, um durch robuste und handwaregestützte Sicherheitsmechanismen insbesondere Schutz vor Bedrohungen wie Seitenkanalangriffen und Speicher Sichersicherheitslücken, welche zu Datenklau und Datenmanipulation führen können.

TEEs, wie etwa Intel SGX, ARM TrustZone oder Sancus, ermöglichen die Isolation und Signierung sicherer Anwendungsbereiche, die als Enklaven bezeichnet werden. /ToDo links

3.Hinweis

Das Prinzip der TEEs erzwingt eine duale Weltansicht der Hardware, welche in eine normale, möglicherweise kompromittierte und einer sicheren und isolierten Welt aufteilt. Dabei haben kompromittierte oder bösartige Systemsoftware in der normalen Welt keinen Zugriff auf den sicheren Speicherbereich der Enklave , welche einen isolierteren Bereich auf dem selben Prozessor und dem selben Adressraum hat. 

4. Hinweis

Dadurch kann die Trusted Computing Base (TCB) reduziert werden. TCB umfasst alle Hardware und Software, die für die sichere Ausführung von vertrauenswürdigen Code notwendig sind. 

Das Ziel ist es, den TCB zu minimieren, da eine größere TCB mehr potenzielle Angriffspunkte bietet und schwerer zu überprüfen und abzusichern ist. Eine kleinere TCB erhöht die Sicherheit, da sie weniger potenzielle Schwachstellen bietet und einfacher zu warten und abzusichern ist. So muss nur der Code, der in der Enklave läuft überprüft und kontrolliert werden.

Dennoch bieten TEEs nur eine grobgranulare Speicherisolation auf Hardware-Ebene und überlassen es den Enklaven- und Applikationsentwicklern, die Sicherheit auf Software-Ebene sicherzustellen. 

Dies führt zu verschiedenen Herausforderungen, da nicht alle dieser Schwachstellen, wie Seitenkanalangriffe, vollständig behebbar sind. Trotz fortwährender Bemühungen von den Entwicklern, werden kontinuierlich neue Schwachstellen identifiziert. Selbst bei erfolgreicher Behebung bestehender Lücken manifestieren sich unaufhörlich neue mögliche  Angriffspunkte, wodurch die Sicherheit der Enklaven gefährdet ist.

In einem solchen Fall könnte die gesamte Enklave kompromittiert werden, was die Sicherheit der gesamten Anwendung gefährdet. Wegen dieser kritischen Schwachstelle kann ein zweiter Ansatz zur Stärkung der Sicherheit notwendig sein, die Kompartimentierung der Software.

8. Hinweis

Gemeint ist dadurch, dass der gesamte Code in TEEs ausgeführt wird und mithilfe einer Softwarelösung wird das Programm aufgeteilt. Anstatt eine einzige TEE zu nutzen, werden mehrere TEEs implementiert. 

Dabei ist es auch möglich verschiedene Sicherheitsstufen der einzelnen Enklaven zu unterstützen. Diese Aufteilung bedeutet, dass ein potenzieller Angreifer mehrere, wenn nicht alle TEEs kompromittieren müsste, um das gesamte System zu übernehmen, was die Angriffskomplexität und das Risiko drastisch senkt.

Da damit aber auch der größte Nachteil der Kompartimentierung auftritt, ist die Notwendigkeit, dass die einzelnen Programmteile nach dem Zero-Trust-Prinzip arbeiten müssen. Das heißt, dass jeder Programmteil als potentiell kompromittiert betrachtet werden muss. 

Diese neuen Angriffsmöglichkeiten, die davor nicht in Betracht gezogen wurden, da das Programm als sicher galt, werden als Compartment Interface Vulnerabilities (CIV) bezeichnet.

Das Paper beschäftigt sich mit den Vorteilen, Problemen und Lösungsmöglichkeiten im Zusammenhang mit fein granularen TEEs. Insbesondere wird analysiert, wie und ob die Sicherheit durch die Implementierung solcher fein granularer Tees gesteigert werden kann, ohne dabei die Komplexität und Angriffsfläche der Anwendung unnötig zu erhöhen, sowie Möglichkeiten zur Minimierung von Sicherheitslücken und des Angriffsbereichs.
