Trusted Execution Environments (TEEs) haben sich als wichtige und notwendige Idee im Bereich der Computersicherheit etabliert, um robuste hardwaregestützte Sicherheitsmechanismen zu bieten. TEEs, wie etwa Intel SGX, ARM TrustZone oder Sancus, ermöglichen die Isolation und Signierung sicherer Anwendungsbereiche, die als Enklaven bezeichnet werden. /ToDo links

Das Prinzip der TEEs erzwingt eine duale Weltansicht, bei der selbst kompromittierte oder bösartige Systemsoftware in der normalen Welt keinen Zugriff auf den Speicherbereich der Enklave hat, welche einen isoliertern Bereich auf dem selben Prozessor und dem selben Adressraum hat. Dadurch kann die Trusted Computing Base (TCB) reduziert werden. Nur der Code, der in der Enklave läuft, muss überprüft, verschlüsselt und kontrolliert werden. Dennoch bieten TEEs nur eine grobgranulare Speicherisolation auf Hardware-Ebene und überlassen es den Enklaven- und Applikationsentwicklern, die Sicherheit auf Software-Ebene sicherzustellen. 

Dadurch entstehen kontinuierlich neue Schwachstellen und Angriffsmöglichkeiten, wie Seitenkanalangriffe, bei denen bereits eine einzige erfolgreiche Attacke ausreicht, um die Isolation und Sicherheit der Enklave zu durchbrechen. In einem solchen Fall könnte die gesamte Enklave kompromittiert werden, was die Sicherheit der gesamten Anwendung gefährdet. Wegen dieser kritischen Schwachstelle kann ein zweiter Ansatz zur Stärkung der Sicherheit notwendig sein, die Kompartimentierung der Software. Gemeint ist dadurch, dass der gesamte Code in TEEs ausgeführt wird und mithilfe einer Softwarelösung wird das Programm aufgeteilt. Anstatt eine einzige TEE zu nutzen, werden mehrere TEEs implementiert. Dabei ist es auch möglich verschiedene Sicherheitsstufen der einzelnen Enklaven zu unterstützen. Diese Aufteilung bedeutet, dass ein potenzieller Angreifer mehrere, wenn nicht alle TEEs kompromittieren müsste, um das gesamte System zu übernehmen, was die Angriffskomplexität und das Risiko drastisch senkt.

Da damit aber auch der größte Nachteil der Kompartimentierung auftritt, ist die Notwendigkeit, dass die einzelnen Programmteile nach dem Zero-Trust-Prinzip arbeiten müssen. Das heißt, dass jeder Programmteil als potentiell kompromittiert betrachtet werden muss. Diese neuen Angriffsmöglichkeiten, die davor nicht in Betracht gezogen wurden, da das Programm als sicher galt, werden als Compartment Interface Vulnerabilities (CIV) bezeichnet.

Das Paper beschäftigt sich mit den Vorteilen, Problemen und Lösungsmöglichkeiten im Zusammenhang mit fein granularen TEEs. Insbesondere wird analysiert, wie und ob die Sicherheit durch die Implementierung solcher fein granularer Tees gesteigert werden kann, ohne dabei die Komplexität und Angriffsfläche der Anwendung unnötig zu erhöhen.
