\section*{Kurzfassung}
Mit dem Aufstieg des Cloud-Computing und der zunehmenden Verlagerung sensibler Daten und kritischer Anwendungen in die Cloud gewinnen Trusted Execution Environments (TEEs) an Bedeutung. Das Hauptziel von TEEs ist die Minimierung der Angriffsfläche und die Isolation von Schwachstellen, um die Integrität und Vertraulichkeit der Daten zu gewährleisten. TEEs bieten eine geschützte Umgebung zur sicheren Verarbeitung und Speicherung sensibler Daten, sind jedoch nicht frei von Schwachstellen. 
Dieses Paper untersucht die Sicherheitslücken feingranularer TEEs und kategorisiert diese in zwei Hauptkategorien: Datenlecks und Datenkorruption. Es wird gezeigt, dass die Schnittstellen zwischen verschiedenen TEEs sowie zwischen sicheren Anwendungen in einer TEE und unsicheren Anwendungen zu erheblichen Sicherheitsrisiken führen können.
Die Ziele dieser Arbeit sind die Darstellung von potenziellen Angriffen und Auswirkungen dieser, um die Sicherheit und Isolation von TEEs zu verbessern. Die Arbeit stützt sich auf die Studien \enquote{Assessing the Impact of Interface Vulnerabilities in Compartmentalized SoftwareAssessing the Impact of Interface Vulnerabilities in Compartmentalized Software}~\cite{CIVPaper} und \enquote{Assessing the Impact of Interface Vulnerabilities in Compartmentalized Software}~\cite{TEEPaper}.