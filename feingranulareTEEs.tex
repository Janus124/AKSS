Unter feingranularen TEEs versteht man das Konzept, nur bestimmte sicherheitskritische Teile eines Programms in isolierten Enklaven auszuführen, anstatt das gesamte Programm in einen gesicherten Kontext zu packen. Dies ermöglicht eine gezielte Absicherung einzelner Komponenten, ohne dass die gesamte Anwendung in TEEs laufen muss.

\subsection{Vorteile feingranularer TEEs}
Hinweis 23
Daraus verspricht man sich mehrere Vorteile. Der offensichtlichste Vorteil ist die Minimierung der Angriffsfläche. Durch die Aufteilung des Programms in mehrere kleinere Kompartimente ist das Programm nicht mehr als Ganzes angreifbar, sondern nur noch die einzelnen, isolierten Kompartimente. Diese Trennung bedeutet, dass selbst wenn ein Angreifer erfolgreich in eine Enklave eindringen kann, der Zugang zu den anderen Programmen erheblich eingeschränkt ist. 

Zusätzlich vertrauen sich die einzelnen Kompartimente nicht mehr Gegenseitig.

Jede Enklave operiert unabhängig und sicher, ohne auf die Integrität der anderen angewiesen zu sein. Durch diese Eigenschaften ermöglicht die feingranulare Struktur außerdem die Isolierung von Schwachstellen. Sicherheitslücken in einem Kompartiment haben keine Auswirkung auf die übrigen Kompartimente. Ein erfolgreicher Angriff auf eine ein einzelnes Kompartiment kann daher keine oder nur wenige Daten freigeben.

Darüber hinaus erlaubt die feingranulare Struktur die Möglichkeit der Implementierung unterschiedlicher Sicherheitsstandards für einzelne Enklaven. Bei der einzelnen Aufteilung des Programms kann statisch ermittelt werden, welche Teile aufgeteilt werden können und welche besonders kritische Funktionen enthalten.
Hinweis 26

Zusammengefasst bieten fein granulare TEEs durch die Reduzierung der Angriffsfläche, die Isolierung von Schwachstellen, die gegenseitige Unabhängigkeit der Kompartimente und die Möglichkeit, spezialisierte Sicherheitsstandards zu implementieren, eine flexible, transparente und robuste Verbesserung der Sicherheit moderner Systeme.