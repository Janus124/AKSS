Die exponentielle Entwicklung im Bereich des Cloud-Computing und die zunehmende Verbreitung von Cloud-Diensten haben die Abhängigkeit von verteilten Computersystemen stark erhöht. Immer mehr Unternehmen und Organisationen verlagern Teile oder ihre gesamten Anwendungen in die Cloud, um von Skalierbarkeit, Flexibilität und Kosteneffizienz zu profitieren. Durch die Nutzung von sensiblen Daten und kritischen Anwendungen gewinnt die Sicherheit dieser Cloud Infrastrukturen eine wichtige Bedeutung, weil sie potentiell die sensiblen Informationen von Millionen von Nutzern und Organisationen verwalten. 

Trusted Execution Environments (TEEs) sind dadurch zu einer wichtiger Technologie geworden, um die Sicherheit sensibler Daten und kritischer Anwendungen zu sichern. TEEs bieten eine sichere und geschützte Umgebung, in der sensible Daten vor potenziellen Bedrohungen geschützt sind. Die Nutzung von Cloud Diensten und die Verlagerung sensibler Daten in große Rechenzentren zeigt, dass TEEs als Schlüsselkomponente in der Datensicherheit zu sehen sind.

Doch trotz ihrer wichtigen Rolle sind TEEs nicht frei von Schwachstellen. Das Grundprinzip der TEEs erzwingt eine duale Weltansicht in den Hardwarekomponenten, welche in eine normale, möglicherweise kompromittierte und eine sichere und isolierte Welt aufgeteilt ist. Dabei können kompromittierte oder bösartige Systemsoftwareteile in der normalen, bzw. kompromittierten Welt keine Daten auf den sicheren Speicherbereich der Enklave lesen, welche einen isolierten Bereich im Adressraum bezeichnet, der aber näher in 2.1 beschrieben wird. Vor allem diese Schnittstelle zwischen den TEEs und zwischen TEE und unsicheren Welt stellt einen potenziellen Angriffsvektor dar. Diese Schwachstelle birgt das Risiko von Datenlecks und Datenkorruption, sowie Seitenkanalangriffen oder Rollbackangriffen. Ein erfolgreiches Ausnutzen von Angreifern könnte nicht nur sensible Informationen kompromittieren, sondern auch die Integrität der gesamten TEE gefährden. 

Dieses Paper legt den Fokus auf die Sicherheitslücken zwischen den Kompartimenten, insbesondere zwischen sicheren Anwendungen innerhalb einer TEE und Anwendungen in der unsicheren Welt. Der Fokus liegt dabei auf Angriffen, die zu Datenlecks und Datenkorruption führen können. (? Zusätzlich wird diskutiert, ob Risiken gemindert und die Sicherheit von TEEs gestärkt werden können)