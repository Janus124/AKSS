Die zunehmende Entwicklung im Bereich des Cloud-Computing und Verbreitung von Cloud-Diensten haben die Nachfrage von verteilten Computersystemen stark erhöht. 
Immer mehr Unternehmen und Organisationen verlagern Teile oder ihre gesamten Anwendungen in die Cloud, mit der Erwartung, von Skalierbarkeit, Flexibilität und Kosteneffizienz zu profitieren~\cite{Cloud}. 
Mit der Nutzung sensibler Daten und kritischer Anwendungen gewinnt die Sicherheit dieser Cloud-Infrastrukturen an Bedeutung, da sie potentiell die sensiblen Informationen von Millionen von Nutzern und Organisationen verwalten.

TEEs sind dadurch zu einer wichtigen Technologie geworden, um die Sicherheit sensibler Daten und kritischer Anwendungen in einer geschützen Umgebung zu gewährleisten. Doch trotz ihrer wichtigen Rolle sind TEEs nicht frei von Schwachstellen. 

Das Grundprinzip der TEEs erzwingt eine duale Weltansicht in den Hardwarekomponenten, welche in eine (1) normale, möglicherweise kompromittierte und eine (2) sichere und isolierte Welt aufgeteilt ist~\cite{TEEPaper}. 
Dabei können kompromittierte oder bösartige Systemsoftwareteile in der normalen, bzw. kompromittierten Welt keine Daten auf dem sicheren Speicherbereich der Enklave, welche einen isolierten Bereich im Adressraum bezeichnet,lesen. In Abschnitt 2.1 wird näher auf Enklaven eingegangen. Vor allem diese, durch die duale Weltansicht entstehende Schnittstelle zwischen verschiedenen TEEs und zwischen TEE und Programmen in der unsicheren Welt, stellen einen potenziellen Angriffsvektor dar.
Diese Schwachstelle birgt das Risiko von Datenlecks und Datenkorruption, sowie Seitenkanalangriffen oder Rollback-Angriffen~\cite{Memory, TEEPaper}. Ein erfolgreiches Ausnutzen dieser Schwachstellen von Angreifern kann nicht nur sensible Informationen kompromittieren, sondern auch die Integrität der gesamten Enklave gefährden. 

Dieses Paper legt den Fokus auf die Sicherheitslücken sowohl zwischen den unterschiedlichen TEEs, als auch zwischen sicheren Anwendungen einer TEE und Anwendungen der unsicheren Welt. Der Fokus liegt dabei auf Angriffen, die zu Datenlecks und Datenkorruption führen können.