Wird die Sicherheit der Daten betrachtet, offenbart sich ein signifikanter Unterschied zwischen Trusted Execution Environments (TEE) mit einem gemeinsamen Adressraum und solchen mit einem separaten. Bei streng separierten Enklaven gestaltet es sich äußerst schwierig für unsichere Prozesse, die TEE zur Freigabe von Daten aus ihrem Speicher freizugeben. Die Konfiguration des Trusted Operating Systems (TOS) kann entweder eine Spiegelung eines Teils des Adressraums oder den Zugriff auf externe Daten für den Datenaustausch in und aus der Enklave ermöglichen. In beiden Fällen ist jedoch ein direkter Zugriff des Programms auf den Speicher der Anwendung nicht mehr gegeben.

Hugo Lefeuvre et al. haben zwei große Kategrorien von Sicherheitslücken identifiziert: Datenlecks und Datenkorruption. Diese Kategorisierung erweist sich als äußerst nützlich, da nahezu alle bekannten Angriffe auf TEEs in eine dieser beiden Kategorien passen. Auch die Angriffsvektoren, die von Jo Van Bulck et al. gefunden und untersucht wurde, lassen sich in diese Kategorien einteilen, was die Gültigkeit dieser Klassifizierung weiter unterstützt. Die Einteiung in Datenlecks und Datenkorruption ermöglicht eine Identifizierung der Angriffsziele und eine Einschätzung der potenziellen Auswirkungen auf die Sicherheit und Integrität der Daten. \cite{TEEPaper} \cite{CIVPaper}

\subsection{Datenlecks}

Die Kategorie der Datenlecks umfasst Angriffe, die darauf abzielen, in irgendeiner Form Daten aus der Enklave zu extrahieren. 

Ein besonders anfälliges Angriffsziel ist der Zustand der TEE beim Verlassen oder Beenden. Unabhängig davon, ob die Beendigung geplant oder durch ein Interrupt erzwungen wird, muss sichergestellt werden, dass keine Daten in einem lesbaren Zustand im Speicher verbleiben. Da das Bereinigen des Speichers eine explizite Softwareaufgabe darstellt, liegt es in der Verantwortung der Entwickler, dies korrekt zu implementieren. Selbst wenn das Bereinigen der Registerzustände in normalen Prozesswechseln üblich ist und automatisch erfolgt, ist diese Schwachstelle nicht zu unterschätzen. Wenn dieser Prozess nicht ordnungsgemäß durchgeführt wird, besteht für Angreifer die Möglichkeit, Registereinträge oder den Stack auszulesen. Dadurch können sie Rückschlüsse auf die Funktionsweise der Anwendung und die letzten Aktionen innerhalb der TEE ziehen.

Die Bedeutung dieser Sicherheitsmaßnahmen wird durch die potenziellen Konsequenzen unzureichender oder falscher Speicherbereinigung unterstützt. Angreifer könnten sensitive Informationen wie kryptografische Schlüssel, Passwörter oder andere vertrauliche Daten extrahieren und somit die Sicherheit der gesamten Anwendung kompromittieren. Daher ist es von entscheidender Bedeutung, dass Entwickler sorgfältig darauf achten, dass alle sicherheitskritischen Daten beim Verlassen der TEE vollständig und sicher gelöscht werden.

Neben dem Verlassen ist die TEE auch während der Laufzeit angreifbar. Konkret ist das über die Parameter möglich, die die TEE von unsicheren Prozessen von außen bekommt. Es ist essenziell, dass diesen Parametern nicht von anfang an vertraut wird. Diese Daten müssen zuerst überpüft und in manchen fällen bereinigt werden. So muss z.B in jedem Fall nicht nur überprüft werden, dass der Inputpointer außerhalb der Enklave liegt, sondern auch, dass der gesammte Speicherbereich des Pointers außerhalb liegt, was je nach Datentyp oder Strukt nicht nur aus einer einfach Überprüfung besteht. Da die größe des Pointers Teil der Eingabe ist und nach dem Angreifermodell davon ausgegangen wird, dass das komplette system möglicherweise komromittiert ist, kann diesem Wert nicht vertraut werden

So ist es z.B möglich, Situationen zu erzeugen in dem die Size wegen Überläufen nicht mehr richtig berechnet wird und so nicht erkannt wird, dass der Datenbereich eigendlich den der Enklave schneidet. Aus diesem Grund ist es dringen notwendig, beim Berechnen von Adressen und beim grundsätzlichen Verwenden von unsicheren Daten, sichere Arithmetik zu verwenden um fehlerhafe Berechnungen zu vermeiden.

Dieser Angriff kann aber Mithilfe der Manipulation von Strings noch weiter geführt werden. Bei der Verwendung einer Low-Level-Sprache wie C verhält sich ein String auch als ein Pointer, welcher auf ein Array zeigt, in dem die Länge mit angegeben wird. Schafft es nun ein Angreifer, z.B. mithilfe der oben beschriebenen Möglichkeit, diesen Datenbereich mit dem der Enklave überschneiden zu lassen, ist es ihm nun Möglich, nahezu den kompletten Speicher auszulesen. Durch die absichtlich falsche Größenangabe und das Weglassen der Nullterminierung, die das Ende eines Strings definiert, kann der Angreifer über das Programm in der TEE auf die Daten innerhalb der Enklave zugreifen. Nun kann der Angreifer über diesen Seitenkanal die Daten auslesen. Ein Seitenkanalangriff meint dabei, dass Rückschlüsse über die Daten anhand von der Ausführzeit oder den Cache Zugriffen gewonnen werden können. \cite{TEEPaper} Wird nun der String ausgelesen kann der Angreifer anhand der Dauer feststellen, wann das nächste Nullbyte im Speicher liegt. Wird dies nun an beliebigen oder auch allen Speicheradressen gemacht, können alle Nullbytes in der Enklave identifiziert werden. 

Je nach Implementation der TEE ist dies anders oder überhaupt Möglich. So ist es z.B. bei Intel SGX-SDK sogar möglich für Angreifer, ein volles Speicherabbild zu generieren. Konkret möglich ist das durch das EDL Attribut, welches das Ende eines Strings definiert. Da dies Standardmäßig das Nullbyte ist, können normalerweise auch nur die Nullbytes ausgelesen werden. Wird dieses EDL Attribut aber mit einer eins überschrieben ist es möglich alle einser in der Enklave zu lokalisieren. Wird dieser Vorgang bis 255 wiederholt, ist es für einen Anwender möglich ein komplettes Speicherabbild zu generieren, was die Sicherheit der TEE komplett untergraben würde. \cite{IntelSGX} \cite{TEEPaper}

Bei diesem Angriff ist aber Anzumerken, dass TEEs nach dem single-World-Prinzip deutlich anfälliger sind, da die Enklave in den speicher des Angreifers eingebettet ist. Bei TEEs nach dem Two-Worlds-Prinzip werden die Daten nur über das TOS geteilt. Da aber in keinem System eine absolute Sicherheit möglich ist, gibt es auch hier Schwachstellen, die von Angreifern ausgenutz werden können.

\subsection{Datenkorruption}

