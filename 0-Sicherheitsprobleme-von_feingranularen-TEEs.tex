Wird die Sicherheit der Daten betrachtet, offenbart sich ein signifikanter Unterschied zwischen TEE mit einem gemeinsamen Adressraum und solchen mit einem separaten. 
Bei streng separierten Enklaven gestaltet es sich äußerst schwierig für unsichere Prozesse, die TEE zur Freigabe von Daten aus ihrem Speicher zu bringen.
Die Konfiguration des TOS kann entweder eine Spiegelung eines Teils des Adressraums oder den Zugriff auf externe Daten für den Datenaustausch in und aus der Enklave ermöglichen. 
In beiden Fällen ist jedoch ein direkter Zugriff des Programms auf den Speicher der Anwendung nicht mehr gegeben.

Hugo Lefeuvre et al.\cite{CIVPaper} haben zwei große Kategorien von Sicherheitslücken identifiziert: Datenlecks und Datenkorruption. Diese Kategorisierung erweist sich als äußerst nützlich, da nahezu alle bekannten Angriffe auf TEEs in eine dieser beiden Kategorien passen. Auch die Angriffsvektoren, die von Jo Van Bulck et al.\cite{TEEPaper} gefunden und untersucht wurde, lassen sich in diese Kategorien einteilen, was die Gültigkeit dieser Klassifizierung weiter unterstützt. Die Einteilung in Datenlecks und Datenkorruption ermöglicht eine Identifizierung der Angriffsziele und eine Einschätzung der potenziellen Auswirkungen auf die Sicherheit und Integrität der Daten.

\subsection{Datenlecks}
Die Kategorie der Datenlecks umfasst Angriffe, die darauf abzielen, Daten aus der Enklave zu extrahieren. 

Ein besonders anfälliges Angriffsziel ist der Zustand der TEE beim Verlassen oder Beenden. Unabhängig davon, ob die Beendigung geplant oder durch ein Interrupt erzwungen wird, muss sichergestellt werden, dass keine Daten in einem lesbaren Zustand im Speicher verbleiben. Da das Bereinigen des Speichers eine explizite Softwareaufgabe darstellt, liegt es in der Verantwortung der Entwickler, dies korrekt zu implementieren. Selbst wenn das Bereinigen der Registerzustände in normalen Prozesswechseln üblich ist und automatisch erfolgt, ist diese Schwachstelle nicht zu unterschätzen. Wird dieser Prozess nicht ordnungsgemäß durchgeführt, können Angreifer Registereinträge oder den Stack auslesen und dadurch Rückschlüsse auf die Funktionsweise der Anwendung und die letzten Aktionen innerhalb der TEE ziehen.

Die Bedeutung dieser Sicherheitsmaßnahmen wird durch die potenziellen Konsequenzen unzureichender oder fehlerhafter Speicherbereinigung verdeutlicht. Angreifer könnten sensible Informationen wie kryptografische Schlüssel, Passwörter oder andere vertrauliche Daten extrahieren und somit die Sicherheit der gesamten Anwendung kompromittieren.

Neben dem Zeitpunkt des Verlassen der TEE, ist die TEE auch während der Laufzeit angreifbar. Konkret ist das über die Parameter möglich, die die TEE von unsicheren Prozessen von außen bekommt. Es ist essenziell, diesen Parametern nicht von Anfang an zu vertrauen. Diese Daten müssen zuerst überprüft und gegebenenfalls bereinigt werden.  Beispielsweise muss überprüft werden, dass der Inputpointer und der gesamte Speicherbereich, auf den er zeigt, außerhalb der Enklave liegen, was je nach Datentyp oder Strukt nicht nur aus einer einfachen Überprüfung besteht. Da die Größe des Pointers Teil der Eingabe ist und nach dem Angreifermodell davon ausgegangen wird, dass das komplette System möglicherweise kompromittiert ist, kann diesem Wert nicht vertraut werden

Eine mögliche Angriffsmethode besteht darin, Situationen zu erzeugen, in denen die Größe wegen Überläufen nicht mehr richtig berechnet werden kann. Dadurch wird nicht erkannt, dass der Datenbereich tatsächlich den der Enklave schneidet. Aus diesem Grund ist es dringend notwendig, beim Berechnen von Adressen und beim grundsätzlichen Verwenden von unsicheren Daten, sichere Arithmetik zu verwenden, um fehlerhafte Berechnungen zu vermeiden.

Dieser Angriff kann aber Mithilfe der Manipulation von Strings noch weitergeführt werden. In Low-Level-Sprachen wie C verhält sich ein String auch als Pointer, der auf ein Array zeigt, wobei die Länge angegeben werden muss. 
Gelingt es einem Angreifer, den Datenbereich mit dem der Enklave überschneiden zu lassen, kann er nahezu den gesamten Speicher auslesen. Durch die absichtlich falsche Größenangabe und das Weglassen der Nullterminierung, die das Ende eines Strings definiert, kann der Angreifer, über das Programm in der TEE, auf die Daten innerhalb der Enklave zugreifen. 
Damit kann der Angreifer über diesen Seitenkanal die Daten auslesen. 
Ein Seitenkanalangriff meint dabei, dass Rückschlüsse über die Daten anhand von der Ausführzeit oder den Cache-Zugriffen gewonnen werden können. Wird nun der String ausgelesen, kann der Angreifer anhand der Dauer feststellen, wann das nächste Nullbyte im Speicher liegt. 
Wird dies nun an beliebigen oder auch allen Speicheradressen gemacht, können alle Nullbytes in der Enklave identifiziert werden\cite{TEEPaper}.



Je nach Implementierung der TEE kann ein Angriff auf diese Art oder anders durchgeführt werden. Beispielsweise ermöglicht das Intel SGX-SDK Angreifern, ein volles Speicherabbild zu generieren. Konkret ermöglicht das EDL-Attribut, welches das Ende eines String definiert, das Auslesen von Speicherinhalten. Standardmäßig wird das Nullbyte verwendet, aber durch das Überschreiben mit anderen Werten ist es möglich, alle entsprechenden Speicherinhalte zu lokalisieren. Wird dieser Vorgang bis 255 wiederholt, kann ein vollständiges Speicherabbild erstellt werden, das die Sicherheit der TEE vollständig untergräbt\cite{IntelSGX, TEEPaper}.

Bei diesem Angriff ist aber anzumerken, dass TEEs nach dem Single-World-Prinzip deutlich anfälliger sind, da die Enklave in den Speicher des Angreifers eingebettet ist. Bei TEEs nach dem Two-Worlds-Prinzip werden die Daten nur über das TOS geteilt. Da aber in keinem System eine absolute Sicherheit möglich ist, gibt es auch hier Schwachstellen, die von Angreifern ausgenutzt werden können.

\subsection{Datenkorruption}
Angriffe dieser Kategorie versuchen die Daten zu überschreiben, um dadurch das Verhalten der TEE zu beeinflussen oder die Integrität zu untergraben.

Eine Möglichkeit, dies zu erreichen, besteht in der Manipulation von Register Flags. Insbesondere die Alignment Check (AC) Flag und die Direction Flag (DF) beeinflussen, wie Daten abgespeichert und gelesen werden. Durch das gezielte Setzen oder Zurücksetzen dieser Flags kann ein Angreifer die TEE dazu bringen, Daten anders als beabsichtigt abzuspeichern, was zu Datenkorruption führen kann.

Auch in diesem Fall bieten Input-Pointer eine große Angriffsfläche. Wenn die Überprüfung der Input-Pointer und die arithmetischen Funktionen nicht korrekt implementiert sind, kann die TEE unabsichtlich ihren eigenen Speicher überschreiben, was die Integrität der Daten zerstört. Während andere Prozesse von außen nicht direkt in die Enklave schreiben können, können Programme innerhalb der Enklave dies sehr wohl. Dies stellt ein erhebliches Risiko dar, da ein kompromittiertes Programm innerhalb der TEE bösartigen Code ausführen und somit die Datenintegrität gefährden kann.

Ein Beispiel für einen solchen Angriff ist der Double-Fetch-Angriff. Hierbei wird das Programm in der TEE dazu gebracht, dieselben Daten mehrfach aus dem Speicher zu lesen. Zwischen den Lesevorgängen können die Daten durch einen Angreifer verändert werden. Dieser Angriff kann auch als Seitenkanalangriff ausgeführt werden, bei dem die Zugriffszeit auf den Speicher genutzt wird, um festzustellen, ob Daten kürzlich verwendet wurden. Wenn die Daten im Cache liegen, ist die Zugriffszeit kürzer, was dem Angreifer Informationen über die Nutzungshäufigkeit und den Speicherort der Daten gibt. Durch diese Methode kann der Angreifer die TEE dazu bringen, gezielt auf manipulierte Adressen zuzugreifen, wodurch die Integrität der Daten weiter kompromittiert wird.

Diese Angriffe verdeutlichen, wie wichtig es ist, robuste Mechanismen zur Überprüfung und Sicherung der Datenintegrität innerhalb der TEE zu implementieren. Nur durch sorgfältige Überprüfung der Input-Pointer, den Einsatz sicherer arithmetischer Funktionen und das Vermeiden von unsicheren Speicherzugriffen kann die Integrität der TEE und ihrer Daten gewährleistet werden.
