Die Trusted Execution Environment (TEE) zeichnet sich durch drei wesentliche Eigenschaften aus, die ihre Funktionalität und Sicherheit beschreiben. Sie gewährleistet (1) die Authentizität des ausgeführten Codes, indem sie sicherstellt, dass dieser tatsächlich von der beabsichtigten Quelle stammt und nicht manipuliert wurde.  Sie sichert (2) die Integrität des Codes, indem sie während der Ausführung sicherstellt, dass dieser nicht verändert wurde und sie schützt (3) die Vertraulichkeit von Code, Daten und Laufzeitvariablen, indem sie sicherstellt, dass diese nur von privilegierten Parteien eingesehen werden kann. 

Aufgrund der Interaktion von Programmen innerhalb der TEE und Programmen außerhalb, kann die TEE auch als Kompartiment betrachtet werden. Diese Struktur ermöglicht eine klare Trennung zwischen den unterschiedlichen Kompartiments und ermöglicht so eine bessere Kontrolle über den Zugriff.

\subsection{TEE Design}
Der genaue Aufbau und das Verhalten von Trusted Execution Environments (TEEs) variieren, lassen sich aber im Allgemeinen in zwei Kategorien einteilen: (1) TEEs, die sich einen Adressraum mit dem unprivilegierten Host teilen, und (2) TEEs, bei denen die CPU konzeptionell in eine normale und eine sichere Welt unterteilt ist.

In der ersten Kategorie erzwingt der Prozessor, dass nur die Enklave selbst auf ihren Speicherplatz zugreifen kann. Eine Enklave ist ein isolierter Speicherbereich innerhalb einer TEE, der dazu dient, Code und Daten vor unautorisiertem Zugriff und Manipulation zu schützen. Selbst privilegierte Benutzer und das Betriebssystem haben keinen Zugriff auf die innerhalb der Enklave gespeicherten Daten. Programme innerhalb der Enklave dürfen jedoch auch auf den Adressraum außerhalb der Enklave zugreifen.Diese Struktur ermöglicht eine flexible Datenverarbeitung, birgt jedoch das Risiko unsicherer Speicherzugriffe, was potenziell zu Sicherheitslücken führen kann.

In der zweiten Kategorie ist der Adressraum strikt in zwei separate Bereiche unterteilt, wobei keiner der beiden Bereiche auf den jeweils anderen zugreifen kann. Der Datenaustausch zwischen diesen beiden Welten erfolgt über das Trusted Operating System (TOS). Da das TOS privilegierten Zugriff hat, kann es einen geteilten Adressraum einrichten, in dem Daten sicher ausgetauscht werden können. Diese Methode bietet eine stärkere Isolation und somit ein höheres Maß an Sicherheit, da direkte Speicherzugriffe zwischen der normalen und der sicheren Welt verhindert werden, jedoch auf Kosten der Datenübertragungsraten.

In beiden Kategorien gewährleistet sowohl der Prozessor, dass extern kein Zugriff auf die Daten möglich ist als auch das TOS, dass der Code in der Enklave sicher ist. Ein weiterer Grundgedanke beim designen einer TEE ist die Annahmen, dass die TEE in einem System ausgeführt wird, das als kompromittiert betrachtet wird und Angreifer die volle Kontrolle über sämtliche Hardware haben.

\subsection{TEE erstellen und verlassen}

Der Enklave Entry/Exit-Prozess spielt eine zentrale Rolle in Trusted Execution Environments (TEEs) und umfasst die Mechanismen ecall (Entry) und ocall (Exit), die die Kontrolle über den Übergang zwischen der normalen und der sicheren Welt, der Enklave, ermöglichen. Einige Implementierungen, wie Intel SGX (!Link), nutzen aber auch die niedrigstufigen Prozessorbefehle eenter und eexit.

Beim Betreten der Enklave erfolgt der Wechsel in die TEE typischerweise über einen ecall. Hierbei wird von der normalen Welt aus eine spezielle Anweisung an das TOS oder, wie im Falle von Intel SGX, an den Prozessor gesendet, um den Eintritt in die Enklave zu initiieren. 

Bevor die Ausführung der Enklaven Anwendung gestartet wird, bereinigt das TOS den Prozessorzustand, um sicherzustellen, dass keine Parameter, Flags oder andere Daten die Funktionsweise der TEE beeinflussen können. Die Parameter werden dabei als untrusted parameters an die CPU weitergegeben, damit diese erst überprüft werden, bevor sie als sicher gelten.

Der Enklave-Exit-Prozess wird durch ocall realisiert. Dies geschieht, wenn die Enklave ihre Ausführung beendet, eine externe Anfrage erhält, die den Übergang in die normale Welt erfordert oder ein Interrupt ausgelöst wird. Durch den Aufruf von ocall wird der Zustand der Enklave gespeichert und die Kontrolle an das TOS zurückgegeben. 

Das TOS bereinigt den Prozessorzustand, um sicherzustellen, dass keine sensiblen Informationen zurückbleiben, bevor der Prozessor aus dem Enklave-Modus in den normalen Betriebsmodus wechselt. Je nach Design werden die Daten für das unsichere OS direkt im Speicher abgelegt oder dafür an das TOS weitergegeben.

Die ecall- und eexit-Mechanismen bieten eine standardisierte Möglichkeit, den Eintritt und Austritt aus Enklaven zu verwalten und gewährleisten eine sichere Ausführung von Anwendungen innerhalb einer Enklave. Diese Mechanismen ermöglichen einen sicheren Übergang zwischen der normalen und der sicheren Welt unter strikter Kontrolle des Trusted Operating Systems.

\subsection{Kompartimentierung}

Die Kompartimentierung ist ein Konzept, das darauf abzielt, ein Computersystem oder einzelne Anwendungen in separate Bereiche aufzuteilen und diese möglichst isoliert voneinander operieren zu lassen. Diese Praxis beruht auf den Prinzipien der Sandbox und der Safebox, die eine sicherere und stabilere Systemumgebung ermöglichen. 

Eine Sandbox ist eine isolierte Ausführungsumgebung, in dem ein Programm ausgeführt wird, welches keinen Zugriff auf Daten anderer Programme hat. Dies gewährleistet nicht nur die Sicherheit sensibler Informationen, sondern verhindert auch die Übernahme von Daten oder die Beeinflussung des Programmes durch potenziell kompromittierte Softwareteile.

Im Gegensatz dazu dient die Safebox dem Schutz der eignen Daten. Dies wird ermöglicht, indem nur privilegierten Prozessen Zugriff auf die Daten gestattet wird. Diese Zugangskontrolle verhindert den Zugriff auf Daten eines anderen Kompartiments. 

Die Kombination aus Sandboxing und Safeboxing ermöglicht eine Trennung zwischen den einzelnen Teilen eines Programmes und sorgt so dafür, dass eine Übernahme des Systems oder ein Zugang zu vertraulichen Daten erschwert wird.

Das Initialisieren eines Kompartments lässt sich in 3 Stufen unterteilen. Zunächst erfolgt die Identifizierung, wo das Programm logisch gut unterteilt werden kann und wie feingranular diese Unterteilung sein soll. Anschließend wird die Durchsetzung der Grenzen vorgenommen, indem die Programme logisch voneinander getrennt und ausschließlich über eine API zur Kommunikation zugelassen werden. 

In der letzten Stufe werden diese Grenzen weiter verstärkt. Hierbei werden die API-Schnittstellen klarer definiert, der Datenfluss möglichst reduziert und Strategien implementiert, wie mit potenziell unsicheren Daten von der unsicheren Welt umgegangen wird.