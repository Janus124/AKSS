Ein wichtiger Begriff im Zusammenhang von TEEs ist die Trusted Computing Base (TCB). Die TCB beschreibt den Bereich eines Programms, der besonders vor externen Angriffen geschützt werden muss, da er mit sensiblen Daten arbeitet oder andere kritische Funktionen übernimmt, wie z. B. das Interagieren mit sensiblen Daten. Wird nun ausschließlich dieser Teil der Anwendung in einer TEE ausgeführt, wird von einer feingranularen TEE gesprochen.

Außerdem können mehrere feingranulare TEEs genutzt werden, indem mehrere kritische Funktionen in separaten TEEs ausgeführt werden, was die Sicherheit weiter erhöhen kann.

\subsection{Minimierung der Angriffsfläche}
Das Hauptziel für die Verwendung von feingranularen TEEs ist die Minimierung der Angriffsfläche und damit die Erhöhung der Sicherheit. 

Programme aus der unsicheren Umgebung haben keinen direkten Zugriff auf Daten, die innerhalb der sicheren Enklave verarbeitet werden. Da feingranulare TEEs in diesem Kontext als Kompartimente behandelt werden können, gibt es für die Kommunikation zwischen sicherer und unsicherer Welt eine definierte Schnittstelle, die durch die als Enklave Description Language (EDL) beschrieben wird. 

Diese Schnittstelle ermöglicht es, genau zu kontrollieren, welche Daten und in welchem Format diese Daten von außen akzeptiert werden. Obwohl diese Schnittstelle die Sicherheit erhöht, stellt sie dafür einen neuen Angriffsvektor dar und führt zu einer neuen Klasse von Sicherheitsproblemen, den Compartment Interface Vulnerabilities (CIVs) \cite{CIVPaper}.

\subsection{Isolierung von Schwachstellen}
Ein weiteres wichtiges Ziel feingranularer TEEs ist ihre Fähigkeit zur Isolierung von Schwachstellen. Ziel ist es, die Kompartimente sicherer zu machen und in der Zukunft die Unabhängigkeit der Kompartimente voneinander zu erreichen.


Es wird immer weiter darauf hingearbeitet, dass Sicherheitslücken in einem Kompartiment keine Auswirkungen auf die anderen Kompartimente haben und so eine Isolation der Sicherheitslücken entsteht. Je mehr Kompartimente eingesetzt werden, desto widerstandsfähiger wird das gesamte System gegen Angriffe. Allerdings steigt mit der Anzahl der feingranularen TEEs auch der Aufwand für die Erstellung, Überprüfung und Verwaltung der Daten sowie der gesamte Overhead. Dennoch kann ein erfolgreicher Angriff auf ein einzelnes Kompartiment möglicherweise nur einen Teil der Daten kompromittieren, während der Rest des Systems sicher bleibt.
