ToDo: Neu machen!!!\\


Dieses Paper basiert hauptsächlich auf den Arbeiten \enquote{Assessing the Impact of Interface Vulnerabilities in Compartmentalized Software}~\cite{CIVPaper} und \enquote{A Tale of Two Worlds: Assessing the Vulnerability of Enclave Shielding Runtimes}~\cite{TEEPaper}. 
Das erstgenannte Paper klassifiziert Sicherheitslücken zwischen Kompartments (CIV) und untersucht, wo diese auftreten. Aufgrund der Art und Weise der Untersuchung, wird aber lediglich die Schnittstelle zwischen Bibliotheken und Anwendung untersucht. 
Alle daraus gewonnenen Rohdaten werden tabellarisch im Paper dargestellt. 
Diese Betrachtung sind für das Verständnis der Sicherheitsprobleme in TEEs von grundlegender Bedeutung.

Das andere Paper stellt zehn direkte Angriffsvektoren vor und erklärt, wie diese sich auf die bekanntesten TEEs auswirken. Hier muss erwähnt werden, dass diese nur theoretisch, mit Codebespiel vorgestellt und nicht mit Zahlen belegt werden.
Die beiden Paper ergänzen sich, indem das erste Kategorien einführt, die vom anderen Paper unterstützt werden. Zusammen bieten sie ein umfassenderes Bild der potenziellen Sicherheitslücken in feingranularen TEEs und tragen damit zur weiteren Entwicklung sicherer TEE-Technologien bei. 
