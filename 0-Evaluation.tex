Dieses Paper basiert hauptsächlich auf den Arbeiten \enquote{Assessing the Impact of Interface Vulnerabilities in Compartmentalized Software}~\cite{CIVPaper} und \enquote{A Tale of Two Worlds: Assessing the Vulnerability of Enclave Shielding Runtimes}~\cite{TEEPaper}. 

Das erstgenannte Paper klassifiziert Sicherheitslücken zwischen Kompartimenten [CIV] und untersucht deren Auftreten. Durch den Einsatz eines Fuzzers, ein automatisiertes Programm zur Indentifikation von Schwachstellen, wird getestet, ob angreifbare Funktionen existieren. Aufgrund der Art und Weise der Untersuchung wird aber lediglich die Schnittstelle zwischen Bibliotheken und Anwendung untersucht. Diese Einschränkung kann jedoch potenzielle Schwachstellen in der Interaktion zwischen anderen Systemkomponenten übersehen.

Im Gegensatz zu dieser praktischen Identifikation von Schwachstellen untersucht das andere Paper die Auswirkung von 10 Angriffen auf die bekanntesten TEEs und beschreibt deren Auftreten. Ein Aspekt dieser Untersuchung ist die theoretische Analyse der Angriffe, die durch konkrete Codebeispiele unterstützt wird. 

Interessanterweise zeigen die in der ersten Arbeit identifizierten kompartimentären Kategorien von Sicherheitslücken eine hohe Übertragbarkeit auf die Angriffsvektoren in TEEs. Dies unterstreicht nicht nur die Relevanz der Kategorien, sondern auch die Möglichkeit, TEEs in diesem Kontext als eine Form von Kompartimenten zu betrachten. Diese Erkenntnis unterstützt die Gültigkeit der in der ersten Arbeit entwickelten Klassifikationen und bietet eine solide Grundlage für die weitere Untersuchung der Sicherheitsprobleme in feingranularen TEEs.
